\documentclass{letter}
\usepackage{hyperref}
\usepackage{amsmath}
\usepackage{amsthm}
\signature{Ricardo Bittencourt}
\begin{document}

\newtheorem{lemma}{Lemma}

\begin{letter}{}
\textbf{1132.} Determine a closed form of the series

  $$ \sum_{n=1}^{\infty} \sum_{m=1}^{\infty}\frac{x^{n+m}}{(n+m)!}$$
    
  \textbf{Solution.} We start by introducing $p=n+m$, so $m=p-n$ and $m\ge 1\implies p\ge n+1$:

  \begin{align*}
    \sum_{n=1}^{\infty} \sum_{m=1}^{\infty}\frac{x^{n+m}}{(n+m)!}
    &= \sum_{n\ge 1} \sum_{m\ge 1}\frac{x^{n+m}}{(n+m)!} \\
    &= \sum_{n\ge 1} \sum_{p\ge n+1}\frac{x^p}{p!} \\
    &= \sum \sum \frac{x^p}{p!}[n\ge 1][p\ge n+1]
  \end{align*}

  Notice that $p\ge n+1\implies n+1\le p \implies n\le p-1$ and now we can interchange the summations.

  \begin{align*}
    \sum \sum \frac{x^p}{p!}[n\ge 1][p\ge n+1]
    &= \sum \sum \frac{x^p}{p!}[1\le n\le p-1] \\
    &= \sum \sum \frac{x^p}{p!}[1\le p-1][n\le p-1] \\
    &= \sum \sum \frac{x^p}{p!}[p\ge 2][1\le n\le p-1] \\
    &= \sum_{p\ge 2} \left(\sum_{1\le n\le p-1} \frac{x^p}{p!}\right)
  \end{align*} 

  This summation can be massaged to bring out the exponential series:

  \begin{align*}
    \sum_{p\ge 2}\left( \sum_{1\le n\le p-1} \frac{x^p}{p!} \right)
    &= \sum_{p\ge 2} \frac{p-1}{p!}x^p \\
    &= \sum_{p\ge 2} \frac{p-1}{p!}x^p + \frac{x^p}{p!} - \frac{x^p}{p!} \\
    &= \sum_{p\ge 2} \frac{p}{p!}x^p - \frac{x^p}{p!} \\
    &= \sum_{p\ge 2} \frac{x^p}{(p-1)!} - \frac{x^p}{p!} \\
    &= \sum_{p\ge 2} x\frac{x^{p-1}}{(p-1)!} -\sum_{p\ge 2} \frac{x^p}{p!} \\
    &= \left(x\sum_{p\ge 1} \frac{x^p}{p!}\right) 
       -\left( -1-x+\sum_{p\ge 0} \frac{x^p}{p} \right) \\
    &= \left(x \left(e^x-1\right)\right) 
       -\left( -1-x+e^x\right) \\
    &= xe^x -x +1 + x-e^x \\
    &= 1+(x-1)e^x
  \end{align*} 

  The desired closed form, therefore, is $1+(x-1)e^x$.

\end{letter}
\end{document}
