\documentclass{letter}
\usepackage{hyperref}
\usepackage{amsmath}
\usepackage{amsthm}
\signature{Ricardo Bittencourt}
\begin{document}

\newtheorem{lemma}{Lemma}

\begin{letter}{}

\textbf{11999.} Evaluate

  $$\sum_{k=1}^{\infty}\frac{(-1)^{\left\lfloor\sqrt{k}+\sqrt{k+1}\right\rfloor}}{k(k+1)}$$

  \textbf{Solution.} Let $k=q^2+r$, where $q,r$ are integers with $q>0$ and $0\le r<2q+1$. We claim that:

  $$(-1)^{\left\lfloor\sqrt{k}+\sqrt{k+1}\right\rfloor} =
    \left.
    \begin{cases}
      1 & \text{for } 0\le r < q \\
      -1 & \text{for } q\le r < 2q+1
    \end{cases}
    \right. $$ 

We need the four lemmas below to prove this.

  \begin{lemma}
    If $0\le r<q$, then $2q < \sqrt{k}+\sqrt{k+1}$.
  \end{lemma}
  \begin{proof}
This is trivial since $\sqrt{q^2+r}+\sqrt{q^2+r+1} > \sqrt{q^2}+\sqrt{q^2}>2q$.
  \end{proof}

  \begin{lemma}
    If $0\le r <q$, then $\sqrt{k}+\sqrt{k+1}<2q+1$.
  \end{lemma}

  \begin{proof}
  We start by noticing that $\left(q+\frac{r}{2q}\right)^2=q^2+r+\frac{r^2}{4q^2}\ge q^2+r$, since the last term is non-negative. Therefore, $q+\frac{r}{2q}\ge\sqrt{q^2+r}$, with equality only when $r=0$. Now we apply it twice:

  \begin{align*}
    \sqrt{q^2+r}+\sqrt{q^2+r+1} 
    &\le \sqrt{q^2+q-1}+\sqrt{q^2+q} \\
    &< q+\frac{q-1}{2q}+q+\frac{q}{2q} \\
    &< 2q+1-\frac{1}{2q} \\
    &< 2q+1 \qedhere
  \end{align*}
  \end{proof}

  \begin{lemma}
    If $q\le r <2q+1$, then $2q+1<\sqrt{k}+\sqrt{k+1}$.
  \end{lemma}
  \begin{proof}
    Consider the sequences $(q,\sqrt{q},1)$ and $(q,\sqrt{q},0)$ and apply Minkowski's inequality:
    \begin{align*}
      \sqrt{q^2+r}+\sqrt{q^2+r+1}  
        &\ge \sqrt{q^2+q}+\sqrt{q^2+q+1}  \\
        &> \sqrt{(2q)^2+(2\sqrt{q})^2+1^2} \\
        &> \sqrt{4q^2+4q+1} \\
        &> 2q+1 \qedhere
    \end{align*}
  \end{proof}

  \begin{lemma}
    If $q\le r <2q+1$, then $\sqrt{k}+\sqrt{k+1}<2q+2$.
  \end{lemma}
\begin{proof}
  We can use again the inequality $\sqrt{q^2+r}\le q+\frac{r}{2q}$ to get:
  \begin{align*}
    \sqrt{q^2+2q}+\sqrt{q^2+2q+1}
    &< q+\frac{2q}{2q}+q+1 \\
    &< 2q+2 \qedhere
  \end{align*}
\end{proof}

When $0\le r<q$, from Lemmas 1 and 2, we have:

\begin{align*}
  2q < \sqrt{k}+\sqrt{k+1} < 2q+1 &\implies 
\left\lfloor \sqrt{k}+\sqrt{k+1} \right\rfloor = 2q \\
  &\implies (-1)^{
  \left\lfloor \sqrt{k}+\sqrt{k+1} \right\rfloor} =1
\end{align*}

When $q\le r<2q+1$, from Lemmas 3 and 4, we have:

\begin{align*}
  2q+1 < \sqrt{k}+\sqrt{k+1} < 2q+2 &\implies 
\left\lfloor \sqrt{k}+\sqrt{k+1} \right\rfloor = 2q+1\\
  &\implies (-1)^{
  \left\lfloor \sqrt{k}+\sqrt{k+1} \right\rfloor} =-1
\end{align*}

Since we can write every positive integer as $q^2+r$ with $0\le r<2q+1$, then we can write the original summation as:

$$\sum_{k=1}^{\infty}\frac{(-1)^{\left\lfloor\sqrt{k}+\sqrt{k+1}\right\rfloor}}{k(k+1)}
=
\sum_{q\ge 1}\left(\sum_{q^2\le r<q^2+q}\frac{1}{r(r+1)}\right)-
\sum_{q\ge 1}\left(\sum_{q^2+q\le r\le q^2+2q}\frac{1}{r(r+1)}\right)
$$

Let's consider the first summation. After splitting the fractions, the sum telescope:

\begin{align*}
\sum_{q\ge 1}\left(\sum_{q^2\le r<q^2+q}\frac{1}{r(r+1)}\right)
  &= 
  \sum_{q\ge 1}\left(\sum_{q^2\le r<q^2+q}\frac{1}{r}-\frac{1}{r+1}\right) \\
  &=
  \sum_{q\ge 1}\left(\frac{1}{q^2}-\frac{1}{q^2+q}\right) \\
  &=
  \sum_{q\ge 1}\left(\frac{1}{q^2}-\frac{1}{q}+\frac{1}{q+1}\right) \\
  &=
  \sum_{q\ge 1}\frac{1}{q^2}
  -\sum_{q\ge 1}\frac{1}{q}
  +\sum_{q\ge 1}\frac{1}{q+1} \\
  &=
  \sum_{q\ge 1}\frac{1}{q^2}
  -\sum_{q\ge 1}\frac{1}{q}
  +\sum_{q\ge 2}\frac{1}{q} \\
  &=
  \sum_{q\ge 1}\frac{1}{q^2}
  -\sum_{q\ge 1}\frac{1}{q}
  +\left(\sum_{q\ge 1}\frac{1}{q}\right) -1\\
  &=
  \sum_{q\ge 1}\frac{1}{q^2}-1 \\
  &=\zeta(2)-1 \\
  &=\frac{\pi^2}{6}-1
\end{align*}

The same can be done with the second summation:

\begin{align*}
\sum_{q\ge 1}\left(\sum_{q^2+q\le r\le q^2+2q}\frac{1}{r(r+1)}\right)
  &= 
  \sum_{q\ge 1}\left(\sum_{q^2+q\le r\le q^2+2q}\frac{1}{r}-\frac{1}{r+1}\right) \\
  &=
  \sum_{q\ge 1}\left(\frac{1}{q^2+q}-\frac{1}{q^2+2q+1}\right) \\
  &=
  \sum_{q\ge 1}\left(\frac{1}{q}-\frac{1}{q+1}-\frac{1}{(q+1)^2}\right) \\
  &=
  \sum_{q\ge 1}\frac{1}{q}
  -\sum_{q\ge 1}\frac{1}{q+1}
  -\sum_{q\ge 1}\frac{1}{(q+1)^2} \\
  &=
  \sum_{q\ge 1}\frac{1}{q^2}
  -\left(\sum_{q\ge 1}\frac{1}{q}\right)+1
  -\left(\sum_{q\ge 1}\frac{1}{q^2}\right)+1 \\
  &=
  2-\sum_{q\ge 1}\frac{1}{q^2} \\
  &=2-\zeta(2) \\
  &=2-\frac{\pi^2}{6}
\end{align*}

The original summation can now be expressed as the two partial sums, arriving the final result:
  
  \begin{align*}
    \sum_{k=1}^{\infty}\frac{(-1)^{\left\lfloor\sqrt{k}+\sqrt{k+1}\right\rfloor}}{k(k+1)}
    &= \frac{\pi^2}{6}-1-\left(2-\frac{\pi^2}{6}\right) \\
    &= \frac{\pi^2}{3}-3
  \end{align*}

\end{letter}
\end{document}
