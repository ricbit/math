\documentclass{letter}
\usepackage{hyperref}
\usepackage{amsmath}
\usepackage{amsthm}
\signature{Ricardo Bittencourt}
\begin{document}

\newtheorem{lemma}{Lemma}

\begin{letter}{}

  \textbf{2154.} Let $f(n)$ denote the number of ordered partitions of a positive integer $n$ such that all of the parts are odd. For example, $f(5)=5$ since 5 can be written as $5$, $3+1+1$, $1+3+1$, $3+1+1$, and $1+1+1+1+1$. Determine $f(n)$.

  \textit{(There is a typo in the problem, $3+1+1$ is repeated, one of them should be $1+1+3$.)} 

  \textbf{Solution.} The generating function for the odd integers is:

  $$ODD(z)=\sum_{n\ge 0}z^{2n+1}=z\sum_{n\ge 0}z^{2n}=\frac{z}{1-z^2}$$

  The ordered partitions of a integer into odd parts can be seen as a non-empty sequence of odd integers. From the theory of analytic combinatorics, a non-empty sequence of objects $SEQ_{\ge 0}(G)$ has a generating function $1/(1-G)-1$. Combining the results, the generating function for the ordered partitions into odd parts is:

  $$F(z)=SEQ_{\ge 0}(ODD(z))=\frac{1}{1-\frac{z}{1-z^2}}-1=\frac{z}{1-z-z^2}$$

  This expression is the well-known generating function of the Fibonacci numbers. Therefore, the solution are the Fibonacci numbers themselves: $f(n)=F_n$.

\end{letter}
\end{document}
