\documentclass{letter}
\usepackage{hyperref}
\usepackage{amsmath}
\usepackage{amsthm}
\signature{Ricardo Bittencourt}
\begin{document}

\newtheorem{lemma}{Lemma}

\begin{letter}{}

\textbf{1258.} For a positive integer $n$, show that

  $$ \left(\frac{
    \cos(1)+\cos(2)+\cos(3)+\cdots+\cos(n)
  }{
    \sin(1)+\sin(2)+\cdots+\sin(n)
  }\right)^2=\frac{2}{1-\cos\left(n+1\right)}-1$$
    
  \textbf{Solution.} Expressing the trigonometric functions as exponentials, we have:

  \begin{align*}
    \left(\frac{
      \cos(1)+\cos(2)+\cos(3)+\cdots+\cos(n)
    }{
      \sin(1)+\sin(2)+\cdots+\sin(n)
    }\right)^2 &=
    \left(\frac{\displaystyle\sum_{1\le x\le n}\frac{e^{i x}+e^{-i x}}{2}}
    {\displaystyle\sum_{1\le x\le n}\frac{e^{i x}-e^{-i x}}{2i}}\right)^2
  \end{align*}

  Each geometric sum can be evaluated in separate, leading to:

  \begin{align*}
    \left(\frac{2i}{2}\right)^2\left(\frac{\displaystyle\sum_{1\le x\le n}e^{i x}+e^{-i x}}
    {\displaystyle\sum_{1\le x\le n}e^{i x}-e^{-i x}}\right)^2
    &=
    -\left(\displaystyle\frac{
        \displaystyle\frac{e^i\left(e^{i n}-1\right) }{e^i-1}+
        \displaystyle\frac{e^{-i n}\left( e^{in}-1\right)}{e^{i}-1}
    }{
        \displaystyle\frac{e^i\left(e^{i n}-1\right) }{e^i-1}-
        \displaystyle\frac{e^{-i n}\left( e^{in}-1\right)}{e^{i}-1}
    }\right)^2 \\
    &=
    -\left(\displaystyle\frac{
      e^i
      +e^{-i n}
    }{
      e^i
      -e^{-i n}
    }
    \right)^2 \\
    &=
    -\left(\displaystyle\frac{
      1
      +e^{-i (n+1)}
    }{
      1
      -e^{-i (n+1)}
    }
    \right)^2 \\
    &=
    -\left(\displaystyle\frac{
      1
      +2e^{-i (n+1)}
      +e^{-2i (n+1)}
    }{
      1
      -2e^{-i (n+1)}
      +e^{-2i (n+1)}
    }
    \right)\\
    &=
    -\left(\displaystyle\frac{
      e^{i(n+1)}
      +2
      +e^{-i (n+1)}
    }{
      e^{i(n+1)}
      -2
      +e^{-i (n+1)}
    }
    \right)
  \end{align*}

  Converting the exponentials back to trigonometric functions:

  \begin{align*}
    -\left(\displaystyle\frac{
      e^{i(n+1)}
      +2
      +e^{-i (n+1)}
    }{
      e^{i(n+1)}
      -2
      +e^{-i (n+1)}
    }
    \right)
    &=
    -\left(\displaystyle\frac{
      2+2\left(\displaystyle\frac{
        e^{i(n+1)} +e^{-i (n+1)}}{2}\right)
    }{
      -2
        +2\left(\displaystyle\frac{e^{i(n+1)} +e^{-i (n+1)}}{2}\right)
    }
    \right)\\
    &=\displaystyle\frac{1+\cos(n+1)}{1-\cos(n+1)}\\
    &=\displaystyle\frac{1+\cos(n+1)}{1-\cos(n+1)}+1-1\\
    &=\left(\displaystyle\frac{1+\cos(n+1)+1-\cos(n+1)}{1-\cos(n+1)}\right)-1\\
    &=\frac{2}{1-\cos(n+1)}-1\\
  \end{align*}
  \qed


\end{letter}
\end{document}
